\documentclass[a4paper]{article}
\usepackage{glossaries}
\usepackage[hidelinks]{hyperref}

\loadglsentries{setup/acronyms.tex}

\title{Summary of the paper}
\date{\today}
\author{Daniel Runge Petersen \and Gustav Graversen \and Lars Emaniel Hansen \and Sebastian Aaholm }

\pagestyle{empty}

\begin{document}
\maketitle
\thispagestyle{empty}

The article discusses the use of \gls{rdf} and \gls{void} in the Semantic Web. It proposes a solution for dynamically generating and updating \gls{void} descriptions for \gls{rdf} datasets to optimize federated queries. It also explains the challenges in keeping \gls{void} descriptions up-to-date and highlights the lack of a standard way to provide approximate numbers for statistics in \gls{void}. It is then described how the tool GraphDB, an \gls{rdf} triple store, was chosen as the database management system due to its ability to use \gls{sparql} for querying data. What precautions were made to ensure the results' validity, and how were the results evaluated.

Using a cache improves the speed and efficiency of the system, with the cache size set to 0 to prevent inaccurate or irrelevant data from being saved. Additionally, it is worth noting that the code for generating and updating \gls{void} descriptions was implemented in Python. Python provides a wide range of libraries and frameworks for working with \gls{rdf} data and enables efficient processing and manipulation of the data.

One dataset was used, the WatDiv dataset containing 10 million triples. The dataset was used for test generating and updating a \gls{void} description. The WatDiv dataset was split into ten smaller datasets, and the initial database creation was made using one split of 1 million tripels. In contrast, the other nine were used to insert data into the database. It then describes how the proposed solution solves these problems and how it can be used to generate and update \gls{void} descriptions for \gls{rdf} datasets.

The text then presents the results of generating and updating \gls{void} descriptions for \gls{rdf} databases. The generation results measured the time it took to generate \gls{void} descriptions for different database sizes. The results showed a linear increase in generation time as the database size increased. However, outliers were observed in the first few runs of the experiment; this was likely due to external factors affecting the measurements. Removing the outliers revealed a stable trend in the generation time.

The update results measured the time it took to update \gls{void} descriptions for different query and database sizes. The query size significantly impacted the update time, while the database size had a minimal effect. The results suggested that as the query size increased, the update time also increased.

A comparison was made between the generation and update times based on the database size. It was observed that generating a \gls{void} description took significantly longer as the database size increased, while the update time remained relatively unaffected by the database size. This indicated that it was more efficient for more extensive databases to update the \gls{void} description rather than generating it from scratch. However, generating the \gls{void} description from scratch was more practical for smaller databases.

Overall, the results highlighted the importance of query size in the update process and suggested avenues for future research, such as investigating the influence of data structure on update times.

\end{document}