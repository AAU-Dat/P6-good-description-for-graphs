
\begin{abstract}
    The \gls{void} can be used to express metadata in the context of \gls{rdf} datasets. Having a dataset described with metadata is useful for humans and machines alike, allowing for better discovery and use of the dataset, but it is a time consuming process to generate and keep this metadata up to date. This paper explores updating metadadata about the statistics of a dataset based on insert operations on the dataset. The insert operations are analyzed using a script, which then creates a set of smaller pre-queries on the dataset, which are then used to generate new metadata. For comparison, the metadata is also generated from scratch on the same dataset. The results show that the metadata can be updated in a reasonable amount of time sometimes, and are just as accurate as when it is generated from scratch. These results are promising in that they highlight a potential way to save time and resources when updating metadata about a dataset, given the right circumstances.
\end{abstract}

\begin{IEEEkeywords}
    Databases, Metadata, RDF, Intelligent Web Services and Semantic Web
\end{IEEEkeywords}