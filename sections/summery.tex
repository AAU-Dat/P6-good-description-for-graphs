\section{Summery}\label{sec:summery}
The article discusses the use of \gls{rdf} and \gls{void} in the Semantic Web. It proposes a solution for dynamically generating and updating \gls{void} descriptions for \gls{rdf} datasets to optimize federated queries. It also explains the challenges in keeping \gls{void} descriptions up-to-date and highlights the lack of a standard way to provide approximate numbers for statistics in \gls{void}. It is then described how the tool GraphDB, an \gls{rdf} triple store, was chosen as the database management system due to its ability to use \gls{sparql} for querying data. What precautions were made to ensure the results' validity, and how were the results evaluated. Using a cache improves the speed and efficiency of the system, with the cache size set to 0 to prevent inaccurate or irrelevant data from being saved. Two datasets were used, a small Pokemon dataset and a larger \gls{watdiv} dataset containing 10 million triples. The smaller dataset was used for testing, while the larger dataset was used to test generating and updating a \gls{void} description from a larger and more complex dataset. The \gls{watdiv} dataset was split into ten smaller datasets, and the initial database creation was made using one split.
In contrast, the other nine were used to insert data into the database. It then describes how the proposed solution solves these problems and how it can be used to generate and update \gls{void} descriptions for \gls{rdf} datasets. The solution is then evaluated using the larger \gls{watdiv} dataset. The results show that the solution is capable of generating and updating \gls{void} descriptions for \gls{rdf} datasets and generating \gls{void} descriptions for large datasets.


