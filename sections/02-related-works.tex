\section{Related Works}\label{sec:related-works}
\emph{Add citations once we add bib file.}

This paper takes inspiration from and expands upon several other ideas. This section describes these ideas and theories.

The primary source of inspiration is the RDF Schema vocabulary VoID, a collection of ways to describe graphs through metadata. This paper presents a new way to cover some of the same areas of metadata that VoID provides to describe RDF graphs but through Hierarchical Data Summarization.

Data summarization is conceptually simple. Similarly to how a whole book may be summarized in a single paragraph, a dataset or database may be presented in a reduced form to convey some information about it that is not simply the data itself. However, unlike a book, data usually evolves over its lifetime, which may also require the related summary to evolve.

The problem of creating and updating data summaries for graph databases is an ongoing effort.

%Citeations
%https://projekter.aau.dk/projekter/files/473989932/P6___Dynamic_Summaries_for_RDF_Graphs.pdf - This project is about generating summaries for RDF graphs. It is a good source of inspiration for the work we are doing. It is also a good source of information about the problems we are trying to solve. Look into their sources for more information.
%https://www.sciencedirect.com/science/article/pii/S1570826811000370
%https://link.springer.com/chapter/10.1007/978-3-319-11955-7_61


