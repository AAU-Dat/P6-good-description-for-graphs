\section{Related Works}\label{sec:related-works}
Data summarization is conceptually simple. Similarly to how a whole book may be summarized in a single paragraph, a dataset or database may be presented in a reduced form to convey some information about it that is not simply the data itself. However, unlike a book, data usually evolves over its lifetime, which may also require the related summary to evolve.

Several other works have been done in data summarization related to our work. However, the following are relevant to our work. Tools such as voiDgen~\cite{creating-void-descriptions} generate \gls{void} descriptions for Web-scale datasets. Furthermore, they employ the MapReduce paradigm to process more significant data sets~\cite{the-mapreduce-paradigm} to compute the \gls{void} content efficiently.

Additionally, there is the Aether Tool~\cite{aether-tool}, which has two main components, the generation of \gls{void} descriptions and a visualization interface. They show off some statistics of a dataset to describe what it contains and show the user how the data is distributed in the dataset. However, the Aether tool's primary focus is on the visualization aspect and describes some of the benefits of visual representation of the data.

The Aether tool is a good source of inspiration for the work we are doing, as the central concept of generating \gls{void} descriptions to help a user determine whether a dataset is relevant for their use is similar to what we want to do, even though their focus seems to lie more in a visual aspect. It is also a good source of information about the problems we are trying to solve.

The problem of creating and updating data summaries for graph databases is an ongoing effort.


%Citeations
%https://projekter.aau.dk/projekter/files/473989932/P6___Dynamic_Summaries_for_RDF_Graphs.pdf - This project is about generating summaries for RDF graphs. It is a good source of inspiration for the work we are doing. It is also a good source of information about the problems we are trying to solve. Look into their sources for more information.
%https://www.sciencedirect.com/science/article/pii/S1570826811000370
%https://link.springer.com/chapter/10.1007/978-3-319-11955-7_61
%https://www.ibm.com/docs/en/netezza?topic=guide-mapreduce-paradigm

