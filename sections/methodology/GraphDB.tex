\section{GraphDB}\lable{sec:graphdb}
This section will cover use of GraphDB, the graph database used in this project. It will also cover the creation of the dataset used in this project.

When choosing which database to use, there were some options we had to consider. We had to decide if we wanted to use a triplestore or a graph database, and what we needed to use the database for etc. There were many popular choices, ~\cite{best_graph_databases} mentions many popular graph databases, and lists some of the pros and cons of each. Such as Neo4j, Stardog, ArangoDB, and GraphDB. We considered using Neo4j, as it was often mentioned in the literature we read, but since it is not a triple store, it is not direclty able to use the SPARQL query language~\cite{neo4j:_a_reasonable_RDF_graph_database}. Instead we would have to use Cypher, Neo4j's graph query language~\cite{cypher_query_language}, or translate a SPARQL query to Cypher.


We chose to use GraphDB, as it is a triplestore that is compliant with W3C standards, and uses the SPARQL query language. As described by the creators of GraphDB: \cite{graphDB}{GraphDB is an enterprise ready Semantic Graph Database, compliant with W3C Standards. Semantic graph databases (also called RDF triplestores) provide the core infrastructure for solutions where modelling agility, data integration, relationship exploration and cross-enterprise data publishing and consumption are important.}
Upon choosing a database to work with, we used docker, to setup a local instance of GraphDB, by doing this, we could easily manage our dataset, and use it for testing purposes.

We then created a dataset, and loaded it into the database. We will cover the creation of the dataset in the next section.


%Notes
%What is the difference between a graph database and a triple store?
%Why we created our own dataset

@online{neo4j:_a_reasonable_RDF_graph_database,
    author  = {Thorsten Liebig},
    url     = {https://neo4j.com/blog/neo4j-rdf-graph-database-reasoning-engine/},
    urldate = {2018-27-02}
}