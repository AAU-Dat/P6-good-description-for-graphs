\section{Void Generation}\label{sec:void}

%Describe what we did with the void generation
%How we did it
%Why we did it
%What we could have done differently
%Describe how this could be useed in practice - with an eye on that this is just theoretical.
This section will cover our implementation of void generation. The section describes how we created a void description based on a dataset, and how we used this to create statistics for the dataset. The section will also cover the thoughts we had about the implementation.

We created a simple python script to generate a void description based on a dataset. This was done to create exact values for statistics of a dataset, and to see if it all worked when changing something in the dataset. The script is simple, and is built from two files. the first file, lib.py, contains all the methods needed to create a void description, and to query our dataset from our local instance of GraphDB. The second file, main.py, is where all the methods are put to use and the void description is created. First the endpoint is set, for our case it was "http://localhost:7200/repositories/pokemon-repository", since we ran the instance locally. Then we created a query to get all data from the dataset from an endpoint. With the endpoint and a query, we create a variable that contains the entire dataset from our endpoint. With the dataset, we can run our method that creates a void description. This method takes the dataset as a parameter, a title, and a short description, the method then returns a void description.

This is how the creation of our \gls{void} description was done. The next section will cover each part of the code, how it works, and why we did it this way.

\subsection{Void Generation Methods}\label{sec:voidmethods}
This section will go into details for each part of the pipeline used to create a void description.

The first method is~\ref{lst:countElements}, this method takes a list as a parameter, and returns the number of elements in the list. This method is used to count the number of triples in the dataset, and is used in the method that creates the void description.

\begin{listing}[htb!]
    \tiny
    \centering
    \begin{minted}{python}
        def CountElements(lst):
            counter = Counter(lst)
            return dict(counter)
    \end{minted}
    \caption{countElements Method}
    \label{lst:countElements}
\end{listing}

The next method is~\ref{lst:countTriples}, this method takes a parsed JSON object as a parameter, and returns the number of triples in the dataset. This method is used to count the number of triples in the dataset, and is used in the method that creates the void description.

\begin{listing}[htb!]
    \tiny
    \centering
    \begin{minted}{python}
# Define a function that counts the number of triples in a parsed JSON object
def CountTriples(parsed_json):
    return len(parsed_json)
    \end{minted}
    \caption{countElements Method}
    \label{lst:countElements}
\end{listing}

The method that combines everything and actually creates the \gls{void} description can be seen in~\ref{lst:VoidCreator}. This method takes 3 parameters as input, a title for the description, a short description, and the dataset as a JSON object. From this

\begin{listing}[htb!]
    \tiny
    \centering
    \begin{minted}{python}
        # Data is the parsed JSON object from the query
        # VoidCreator takes the parsed JSON object and creates a VOID description from it, that details the number of triples, distinct subjects, distinct objects, and distinct properties in the dataset
def VoidCreator(title, description, data):
    data = data["results"]["bindings"]
    subjects_dictionary = CreateUniqueOccurenceCountDictionaryionary(data, "s")
    predicate_dictionary = CreateUniqueOccurenceCountDictionaryionary(
        data, "p")
    object_dictionary = CreateUniqueOccurenceCountDictionaryionary(data, "o")

    return CreateBaseVoidDescription(title, description, CountTriples(data),
                                     len(subjects_dictionary),
                                     len(object_dictionary),
                                     len(predicate_dictionary))
    \end{minted}
    \caption{Void Generation Methods}
    \label{lst:VoidCreator}
\end{listing}


% \begin{listing}[htb!]
%     \tiny
%     \centering
%     \begin{minted}{python}
%     \end{minted}
%     \caption{}
%     \label{}
% \end{listing}