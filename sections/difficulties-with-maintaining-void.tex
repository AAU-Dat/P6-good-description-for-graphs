\section{Difficulties with evolving ontologies}\label{sec:difficulties-with-evolving-ontologies}

The ontology of a dataset can change over time, this can be due to the addition of new data, the addition of new types of data, or simply the meaning of some of the data changing. But handeling these changes is not a trivial task, especially since the aim of this paper is to automate the process of updating the \gls{void} description of a dataset. This section will discuss the difficulties that arise when trying to update the \gls{void} description of a dataset when the ontology of the dataset changes~\cite{evolving-ontology-evolution}.

\subsection{Ontology evolution}
To better graps the concept of \gls{rdf}, \gls{void}, and ontologies, it was decided to create a small dataset that could be worked on. This dataset was a graph of Pokemon, where each Pokemon had a type, and some had evolutions. This dataset was created by hand, and was used to test the different methods of updating the \gls{void} description of a dataset.




%https://www.mecs-press.org/ijmecs/ijmecs-v9-n3/IJMECS-V9-N3-7.pdf - Main focus on 'dead links' and 'broken links' - not really relevant to our project
%https://link.springer.com/chapter/10.1007/978-3-540-70960-2_2 - Seems to highlight issues with ontology evolution
%https://link.springer.com/article/10.1007/s13740-013-0027-z - Ontology evolution


