\section{Introduction}\label{sec:introduction2}
\task{Add IEEEPARstart command to start of section here. Looks good, even though it's not a journal paper}
% \IEEEPARstart{A}{}

\gls{rdf} is a fundamental technology for the Semantic Web, providing a standardized approach for representing data as graph-based structures \cite{the-web-of-data}. The \gls{void} is a valuable complement by providing a framework for describing the metadata of RDF datasets. However, the \gls{void} descriptions can quickly become outdated due to databases often needing to be updated and changed. This can lead to problems when querying \gls{rdf} data, as the \gls{void} descriptions can no longer be used to optimize the queries or can potentially lead to incorrect results.

\gls{void} is a vocabulary for describing \gls{rdf} datasets that expresses metadata about the datasets. As a result, this helps users understand the contents of a database and find the correct data needed for a given situation, such as the compatible dataset format. \gls{void} has features to describe the format of the data clearly. This can be used to optimize queries, as the \gls{void} description can be used to determine which parts of the database are relevant to the query. \task{Add cite to https://www.w3.org/TR/2011/NOTE-void-20110303/}

Where \gls{void} fails is in the case when a new kind of edge is added. In a graph of data, it is trivial to add an edge between two nodes, however in semantic web (or \gls{rdf}), if this type of edge is the first of its kind, this update represents an update to the ontology of the data which now needs to be reflected in the ontology AND in the \gls{void} description. This is a problem that is not unique to \gls{void}, but is a problem that is present in any system that is based on a static versus dynamic ontology.
\question{Is RDF concerned with generating meaning from structure, or structure from meaning? Reading about the semantic web, it seems that it is concerned with both, and that is not possible.}

\cite{creating-void-descriptions} introduces tools that can be used to generate \gls{void} descriptions for \gls{rdf} datasets. Although it is not intended to replace the manual creation of \gls{void} descriptions completely, it can be used to generate \gls{void} descriptions for large datasets, which can then be manually edited to ensure that the \gls{void} descriptions are correct. However, it seems that this approach is meant to be used once for a dataset. Then the \gls{void} description is manually edited in the future to ensure that it is correct, or it might indicate that the tool is used on a dataset when a significant change has happened. Finally, the \gls{void} description is re-generated. This means that the \gls{void} descriptions are not dynamically updated when the dataset is updated, which could lead to issues.\task{We have to double check the source if they mention anything about this. They do not directly say if it is only generated once.}

This paper proposes an approach for dynamically generating, and updating \gls{void} descriptions for \gls{rdf} datasets. This problem is similar to graph summarization, where the graph is updated and the summary is updated to reflect the changes in the graph. However, the problem is more complex, as the \gls{rdf} graph is not simply a representation of data, but also meaning. Generated \gls{void} descriptions can be usefull as they might ensure that all the relevant information is included, and that descriptions follow a simillar pattern, for ease of use, hereby reducing the need for manual editing. Furthermore, with the dynamic generation, \gls{void} descriptions can be kept up to date and can, as a result of this, be used to optimize federated queries.




% One of the challenges in utilizing RDF data is the sheer volume of data available on the web, which is often distributed across different organizations and locations. Federated querying is a powerful technique for querying RDF data across multiple sources, enabling efficient integration and data exchange. However, federated querying requires a detailed understanding of the metadata of the RDF datasets, including their structure, content, and interrelationships. \cite{Cheng og Harting}

%The Vocabulary of Interlinked Datasets (VOID) has been developed to address this challenge. VOID provides a standardized way of describing the metadata of RDF datasets, enabling the optimization of federated queries. By using VOID, data consumers can understand the RDF datasets they are querying, including the types of data, their relationships, and the cardinality of their attributes.

%%%%%%%%
%https://www.w3.org/TR/2011/NOTE-void-20110303/ - VOID description
%https://metacpan.org/pod/RDF::Generator::Void - A perl module for generating VOID descriptions.
%https://www.sciencedirect.com/science/article/pii/S1570826811000370 - Creating VOID descriptions for RDF datasets.
%https://www.hpi.uni-potsdam.de/naumann/sites/btc2010/ - A paper on creating VOID descriptions for RDF datasets.
