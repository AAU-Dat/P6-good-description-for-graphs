\section{Introduction}\label{sec:introduction2}
\IEEEPARstart{A}{} fundamental technology for the Semantic Web is \gls{rdf}, providing a standardized approach for representing data as graph-based structures. The \gls{void} is a valuable complement by providing a framework for describing the metadata of \gls{rdf} datasets. However, the \gls{void} descriptions can quickly become outdated due to databases often needing to be updated and changed. This can lead to problems when querying \gls{rdf} data, as the \gls{void} descriptions can no longer be used to optimize the queries or can potentially lead to incorrect results.

\gls{void} is a vocabulary for describing \gls{rdf} datasets that expresses metadata about the datasets. As a result, this helps users understand the contents of a database and find the correct data needed for a given situation, such as the compatible dataset format. In addition, \gls{void} has features to describe the data format clearly. This can be used to optimize queries, as the \gls{void} description can be used to determine which parts of the database are relevant to the query~\cite{documentation-void}.

Where \gls{void} fails is in the case when a new kind of edge is added. For example, in a graph of data, it is trivial to add an edge between two nodes, however in the semantic web (or \gls{rdf}), if this type of edge is the first of its kind, this update represents an update to the ontology of the data which now needs to be reflected in the ontology and the \gls{void} description. This is a problem that is not unique to \gls{void} but is a problem that is present in any system that is based on a static versus dynamic ontology.

\cite{creating-void-descriptions} introduces tools that can be used to generate \gls{void} descriptions for \gls{rdf} datasets. Although it is not intended to replace the manual creation of \gls{void} descriptions completely, it can be used to generate \gls{void} descriptions for large datasets, which can then be manually edited to ensure that the \gls{void} descriptions are correct. However, this approach is meant to be used once for a dataset. Then the \gls{void} description is manually edited in the future to ensure that it is correct, or it might indicate that the tool is used on a dataset when a significant change has happened. Finally, the \gls{void} description is re-generated. Unfortunately, this means that the \gls{void} descriptions are not dynamically updated when the dataset is updated, which could lead to issues. \task{We have to double-check the source if they mention anything about this. They do not directly say if it is only generated once.}

This paper proposes an approach for dynamically generating and updating \gls{void} descriptions for \gls{rdf} datasets. This problem is similar to graph summarization, where the graph is updated, and the summary is updated to reflect the changes in the graph. However, the problem is more complex, as the \gls{rdf} graph is not simply a data representation but also meaning. Nevertheless, generated \gls{void} descriptions can be helpful as they ensure that all the relevant information is included and that descriptions follow a similar pattern for ease of use, thereby reducing the need for manual editing. Furthermore, with dynamic generation, \gls{void} descriptions can be kept up to date and can, as a result of this, be used to optimize federated queries.

Federated queries are queries that are executed on multiple datasets~\cite{intro-federated-query}. This is useful when the data needed for a query is spread across multiple datasets. However, federated queries require a detailed understanding of the metadata of the \gls{rdf} datasets, including their structure, content, and interrelationships. \gls{void} descriptions can be used to optimize federated queries by providing the necessary metadata~\cite{rdf-federated-query}.

\subsection{Approximations in VoID} \label{sec:approximations}
The \gls{void} documentation~\cite{documentation-void} states that approximate numbers can be provided for the statistics in \gls{void}.
However, there is no standard way of doing this. Therefore it is up to the user to decide how to do this. For example, the user can manually calculate and insert the approximate number or use a tool~\cite{the-web-of-data}.

The advantages of estimations is that it is faster to calculate and can be used to estimate the size of the dataset. However, the estimations can be wrong, and the user cannot be sure that the estimations are correct. Therefore it is better to use the exact numbers instead of estimations.

The tool can be a program that calculates or uses a sampling method to estimate the number. However, the sampling method is the most common way of doing this; one way of calculating the statistics on an \gls{rdf} dataset can be seen here~\cite{zneika2016rdf}.

The user can also use the exact number because it is the most precise and easy to calculate the statistics. That way, the user can be sure that the statistics are correct and can check the statistics with a query on the dataset. Moreover, when a query can easily find the precise number, it is unnecessary to use an approximation because the approximation will not be more precise than the exact number.
This is also why it was chosen in the project to use the exact numbers instead of approximations in the  \gls{void} description of the dataset in the project. When the dataset is updated, the statistics will be updated as well, and the user can be sure that the statistics are correct and not approximations that may not be correct. One reason behind using the exact values is that when changes occur in the dataset, the metadata statistics will also be updated. This is not the case when using approximations. Therefore, the exact values are used instead of approximations.


% One of the challenges in utilizing RDF data is the sheer volume of data available on the web, which is often distributed across different organizations and locations. Federated querying is a powerful technique for querying RDF data across multiple sources, enabling efficient integration and data exchange. However, federated querying requires a detailed understanding of the metadata of the RDF datasets, including their structure, content, and interrelationships. \cite{Cheng og Harting}

%The Vocabulary of Interlinked Datasets (VOID) has been developed to address this challenge. VOID provides a standardized way of describing the metadata of RDF datasets, enabling the optimization of federated queries. By using VOID, data consumers can understand the RDF datasets they are querying, including the types of data, their relationships, and the cardinality of their attributes.

%%%%%%%%
%https://metacpan.org/pod/RDF::Generator::Void - A perl module for generating VOID descriptions.
%https://www.hpi.uni-potsdam.de/naumann/sites/btc2010/ - A paper on creating VOID descriptions for RDF datasets.
