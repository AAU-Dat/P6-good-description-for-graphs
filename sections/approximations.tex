\subsection{Approximations in VoID} \label{sec:approximations}
By the VoID  documentation, it is stated that approximate numbers can be provided for the statistics in VoID. However there is not standard way of doing this, therefore it is up to the implementer to decide how to do this. The user can manully do the calculation and insert the approximate number, or the user can use a tool to do this~\cite{the-web-of-data}. The tool can be a program that does the calculation, or a tool that uses a sampling method to estimate the number. The sampling method is the most common way of doing this, one way of calculating the statistics on a RDF dataset can be seen here~\cite{zneika2016rd}.

@inproceedings{zneika2016rdf,
    title={Rdf graph summarization based on approximate patterns},
    author={Zneika, Mussab and Lucchese, Claudio and Vodislav, Dan and Kotzinos, Dimitris},
    booktitle={Information Search, Integration, and Personalization: 10th International Workshop, ISIP 2015, Grand Forks, ND, USA, October 1-2, 2015, Revised Selected Papers 10},
    pages={69--87},
    year={2016},
    organization={Springer}
}