\section{Introduction}\label{sec:introduction2}
\emph{Dynamic generation of VoID descriptions for RDF (VoID metadata)}

RDF is a core technology of the Semantic Web, offering a standard and adaptable approach to representing data as graph-based structures \emph{the web of data}. The Vocabulary of Interlinked Datasets (VOID) serves as a valuable complement by providing a framework for describing the metadata of RDF datasets \emph{Cheng og Harting}. But due to databases often being updated and changed, the VOID descriptions can quickly become outdated. This can lead to problems when querying RDF data, as the VOID descriptions can no longer be used to optimize the queries, or can potentially lead to incorrect results.

\emph{Creating VOID descriptions for Web-scale Data} introduces tools that can be used to generate VOID descriptions for RDF datasets. Altough it is not intended to complety replace the manual creation of VOID descriptions, it can be used to generate VOID descriptions for large datasets, which can then be manually edited to ensure that the VOID descriptions are correct. But it seems that this approach is meant to be used once for a dataset, and then the VOID description is manually edited in the future to ensure that it is correct, or it might indicate that the tool is used on a dataset when a significant change has happened, and the VOID description is simply re-generated \todo{We have to double check the source if they mention anything about this}. This means that the VOID descriptions are not dynamically updated when the dataset is updated, which could lead to issues.

This paper proposes an approach for dynamically generating VOID descriptions for RDF datasets. Our approach enables the automatic creation of VOID descriptions. With dynamic generation, VOID descriptions can be kept up to date, and can hereby be used to optimize federated queries. 



% One of the challenges in utilizing RDF data is the sheer volume of data available on the web, which is often distributed across different organizations and locations. Federated querying is a powerful technique for querying RDF data across multiple sources, enabling efficient integration and data exchange. However, federated querying requires a detailed understanding of the metadata of the RDF datasets, including their structure, content, and interrelationships. \cite{Cheng og Harting}

% The Vocabulary of Interlinked Datasets (VOID) has been developed to address this challenge. VOID provides a standardized way of describing the metadata of RDF datasets, enabling the optimization of federated queries. By using VOID, data consumers can understand the RDF datasets they are querying, including the types of data, their relationships, and the cardinality of their attributes.

%%%%%%%%

%https://metacpan.org/pod/RDF::Generator::Void - A perl module for generating VOID descriptions.
%https://www.sciencedirect.com/science/article/pii/S1570826811000370 - Creating VOID descriptions for RDF datasets.
    %https://www.hpi.uni-potsdam.de/naumann/sites/btc2010/ - A paper on creating VOID descriptions for RDF datasets.

% Citations
% @book{the web of data,
% author = {Hogan, Aidan},
% address = {Cham, Switzerland},
% booktitle = {The web of data},
% edition = {1st ed. 2020.},
% isbn = {3-030-51580-X},
% keywords = {Semantic Web},
% language = {eng},
% publisher = {Springer},
% title = {The web of data },
% year = {2020},
% }

% @book{RDF database systems,
% author = {Curé, Olivier, author.},
% address = {Waltham, Massachusetts :},
% booktitle = {RDF database systems : triples storage and SPARQL query processing /},
% edition = {First edition.},
% isbn = {0-12-799957-4},
% publisher = {Morgan Kaufmann,},
% title = {RDF and the Semantic Web Stack},
% year = {2015},
% }




