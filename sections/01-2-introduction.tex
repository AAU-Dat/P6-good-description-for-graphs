\section{Introduction}\label{sec:introduction2}
\emph{Dynamic generation of VoID descriptions for RDF (VoID metadata)}

RDF is a core technology of the Semantic Web, offering a standard and adaptable approach to representing data as graph-based structures \emph{the web of data}. The Vocabulary of Interlinked Datasets (VOID) serves as a valuable complement by providing a framework for describing the metadata of RDF datasets \emph{Cheng og Harting}. But due to databases always being updated and changed, the VOID descriptions can quickly become outdated. This can lead to problems when querying RDF data, as the VOID descriptions can no longer be used to optimize the queries, or can potentially lead to incorrect results.

\emph{Creating VOID descriptions for Web-scale Data} introduces tools that can be used to generate VOID descriptions for RDF datasets. Altough it is not intended to complety replace the manual creation of VOID descriptions, it can be used to generate VOID descriptions for large datasets, which can then be manually edited to ensure that the VOID descriptions are correct. But this seems (Lars: be sure of this) that this approach is meant to be used once for a dataset, and then the VOID description is manually edited to ensure that it is correct, or that the tools are to be used multiple times, intead of simply running constantly. This means that the VOID descriptions are not updated when the dataset is updated, which could lead to issues.

This paper proposes an approach for dynamically generating VOID descriptions for RDF datasets. Our approach enables the automatic creation of VOID descriptions, to describe the metadata of an RDF dataset, including information on the types of data, and their relationships. By dynamically generating VOID descriptions, VOID descriptions can be kept up to date, and can hereby be used to optimize federated queries. 



% One of the challenges in utilizing RDF data is the sheer volume of data available on the web, which is often distributed across different organizations and locations. Federated querying is a powerful technique for querying RDF data across multiple sources, enabling efficient integration and data exchange. However, federated querying requires a detailed understanding of the metadata of the RDF datasets, including their structure, content, and interrelationships. \cite{Cheng og Harting}

% The Vocabulary of Interlinked Datasets (VOID) has been developed to address this challenge. VOID provides a standardized way of describing the metadata of RDF datasets, enabling the optimization of federated queries. By using VOID, data consumers can understand the RDF datasets they are querying, including the types of data, their relationships, and the cardinality of their attributes.











%https://metacpan.org/pod/RDF::Generator::Void - A perl module for generating VOID descriptions.
%https://www.sciencedirect.com/science/article/pii/S1570826811000370 - Creating VOID descriptions for RDF datasets.
    %https://www.hpi.uni-potsdam.de/naumann/sites/btc2010/ - A paper on creating VOID descriptions for RDF datasets.

% Citations
% @book{the web of data,
% author = {Hogan, Aidan},
% address = {Cham, Switzerland},
% booktitle = {The web of data},
% edition = {1st ed. 2020.},
% isbn = {3-030-51580-X},
% keywords = {Semantic Web},
% language = {eng},
% publisher = {Springer},
% title = {The web of data },
% year = {2020},
% }

% @book{RDF database systems,
% author = {Curé, Olivier, author.},
% address = {Waltham, Massachusetts :},
% booktitle = {RDF database systems : triples storage and SPARQL query processing /},
% edition = {First edition.},
% isbn = {0-12-799957-4},
% publisher = {Morgan Kaufmann,},
% title = {RDF and the Semantic Web Stack},
% year = {2015},
% }





%%%%%%%% Notes
We would like to describe RDF endpoints.
We would like to describe it in a manner that can effectively be used to describe federated queries.
We would like to make it simple for a machine to understand what a dataset contains, to make it easier for them to query relevant data.
We would like it to be used by a publisher, to describe the dataset in a way that makes it possible for a user to use a machine to extract important information from the data for the machine to make descicions about whether or not to construct a query, and how that query should look.    
----
How do we handle changes in the data? How do we handle changes in the voiD description?
How do ensure that all publishers live up to a common standard?
    - Should that be auto generated, used on a database?


----    
Find different IRI standards(for RDF datasets): Lars will add this
    -
Types of RDF endpoints: Lars will add this
    - SPARQL endpoint
        - Mentioned often

Difficulties in federated queries for RDF data: Gustav will add this
    - 

Compare other approaches for RDF data: Sebastian will add this
    - 



%For now, it seems that voiD(Vocabulary of Interlinked Datasets) is a good option for us to work with RDF datasets. 

% here is a link to the voiD description: http://www.w3.org/TR/void/
% https://ceur-ws.org/Vol-538/ldow2009_paper20.pdf




%%%%%%%%%%%%%%%%%%%%%%%%%%%%%


RDF is a core technology of the Semantic Web stack, offering a standard and adaptable approach to representing data as graph-based structures. To enhance the effectiveness of RDF, the Vocabulary of Interlinked Datasets (VOID) serves as a valuable complement by providing a framework for describing the metadata of RDF datasets, encompassing their structure, content, and interrelationships. With its ability to facilitate federated queries, VOID plays a crucial role in optimizing the use of RDF data in the Semantic Web, enabling efficient integration and exchange of data.

In this paper, we propose a novel approach to improve federated queries through the use of hierarchical data summaries and graphs described by other graphs. Our approach involves the creation of meta RDF data, which summarizes the structure and content of the RDF datasets being queried. We demonstrate how this meta data can be used to optimize federated queries, by enabling more efficient routing of queries and reducing the amount of data that needs to be transmitted between data sources.

We also present a technique for representing graphs described by other graphs, which allows us to capture complex relationships between RDF datasets. This technique enables us to incorporate information about the structure and content of the RDF datasets into our meta RDF data, thus providing a more comprehensive view of the data being queried.

%%%%%%%%%%%%%%%%%%%%%%%%%%%%%%% #1

RDF is a fundamental technology in the Semantic Web, providing a standardized approach to represent data as graph-based structures. An emerging concept in this field is using graphs described by other graphs, which capture complex relationships between RDF datasets. This concept enhances the ability to represent and query complex data structures in a distributed environment, providing a more comprehensive view of the data being queried and enhancing its potential applications in diverse domains.

Hierarchical data summaries provide a way to represent significant and complex datasets in a condensed form by aggregating information at different levels of granularity. Our approach leverages these summaries to create meta RDF data that summarizes the structure and content of the queried RDF datasets. Using such meta RDF data enables more efficient routing of queries and minimizes data transmission between sources, resulting in faster query response times and reduced network traffic. Additionally, we present a technique to represent graphs described by other graphs, capturing complex relationships between RDF datasets and providing a more comprehensive view of the data being queried.

However, optimizing the use of RDF data in the Semantic Web can be challenging due to the data's distributed nature and the data's heterogeneity. To address these challenges, the Vocabulary of Interlinked Datasets (VOID) provides a valuable framework for describing the metadata of RDF datasets, including their structure, content, and interrelationships. Furthermore, with its ability to facilitate federated queries, VOID enables efficient integration and data exchange on the Semantic Web. 

In this paper, we propose using hierarchical data summaries and graphs described by other graphs has significant practical implications, particularly in query optimization and integration of RDF data in the Semantic Web. Our approach can address the challenges in federated queries by enabling more efficient routing of queries and minimizing data transmission between sources. This results in faster query response times and reduced network traffic, which is critical in optimizing the performance of the Semantic Web

%%%% #2
RDF is a fundamental technology for the Semantic Web, providing a standardized approach for representing data as graph-based structures. However, optimizing the use of RDF data in the Semantic Web can be challenging due to the distributed nature of the data and the heterogeneity of the data sources. To address these challenges, the Vocabulary of Interlinked Datasets (VOID) provides a valuable framework for describing the metadata of RDF datasets, including their structure, content, and interrelationships. With its ability to facilitate federated queries, VOID plays a critical role in enabling efficient integration and exchange of data in the Semantic Web. 

The motivation of this project is to enhance the efficiency and effectiveness of federated queries in the Semantic Web by utilizing hierarchical data summaries and graphs described by other graphs. This is achieved by creating meta RDF data that summarizes the structure and content of the RDF datasets being queried. By representing complex relationships between RDF datasets through graphs described by other graphs, our approach provides a more comprehensive view of the data being queried. The practical implications of our proposed approach are significant, as it has the potential to accelerate innovation and enable new insights in a range of domains, including e-commerce and scientific research, by optimizing the integration and exchange of data across multiple domains and organizations in the Semantic Web.
In this paper, we present a novel approach to improving federated queries using hierarchical data summaries and graphs described by other graphs. Our approach involves the creation of meta RDF data that summarizes the structure and content of the RDF datasets being queried. This meta data can be used to optimize federated queries by enabling more efficient routing of queries and reducing the amount of data transmitted between data sources. By representing graphs described by other graphs, our technique captures complex relationships between RDF datasets, providing a more comprehensive view of the data being queried. 

The proposed approach has significant practical implications, as it addresses critical challenges in querying and integrating RDF data in the Semantic Web. The ability to optimize federated queries through the use of meta RDF data can enhance the effectiveness of the Semantic Web in a range of domains, from e-commerce to scientific research. By enabling efficient integration and exchange of data across multiple domains and organizations, our approach has the potential to accelerate innovation and enable new insights in the Semantic Web.