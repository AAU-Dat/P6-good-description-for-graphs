\section{Methodology}\label{sec:methodology}
\task{Consider title. Method might be better than Methodology}

% Dynamic generation of VoID descriptions for RDF (VoID metadata)

Where \gls{void} fails is in the case when a new kind of edge is added. In a graph of data, it is trivial to add an edge between two nodes, however in semantic web (or RDF), if this type of edge is the first of its kind, this update represents an update to the ontology of the data which now needs to be reflected in the ontology AND in the \gls{void} description. This is a problem that is not unique to \gls{void}, but is a problem that is present in any system that is based on a static versus dynamic ontology.

\question{Is RDF concerned with generating meaning from structure, or structure from meaning? Reading about the semantic web, it seems that it is concerned with both, and that is not possible.}

This project deals with updating the \gls{void} description of a dataset when the dataset is updated. This problem is similar to graph summarization, where the graph is updated and the summary is updated to reflect the changes in the graph. However, the problem is more complex, as the \gls{rdf} graph is not simply a representation of data, but also meaning.