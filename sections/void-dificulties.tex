\subsection{Difficulties in Updating VoID Descriptions in RDF Databases} \label{sec:difficulties-void}
Updating VoID descriptions in an RDF database can be difficult for several reasons, for one the lack of standardization. VoID descriptions are metadata that describe the structure of an RDF dataset, and there is no standard way to represent VoID descriptions. As a result, different datasets may use different formats or conventions to represent VoID descriptions, which can make it difficult to update them consistently~\cite{documentation-void}. Second the complexity of RDF data. RDF data can be very complex, with multiple levels of nested statements and complex relationships between entities. This complexity can make it difficult to identify which parts of the data need to be updated and how to update them without introducing errors~\cite{the-web-of-data}. Third the lack of tooling. There are relatively few tools available for working with VoID descriptions in RDF databases, and many of them are specialized or require a high level of technical expertise to use effectively. This can make it difficult for non-experts to update VoID descriptions in a way that is accurate and efficient~\cite{mendes2019voidext}. Finally the size of the dataset. Updating VoID descriptions in a large RDF dataset can be time-consuming and resource-intensive, especially if the updates involve significant changes to the structure of the data. This can make it difficult to keep VoID descriptions up-to-date and accurate over time~\cite{the-web-of-data}.
