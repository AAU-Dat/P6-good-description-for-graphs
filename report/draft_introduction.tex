There are many ways to represent data in the Semantic Web. One interesting way is to use RDF datasets. RDF data is usually stored in a triple store, which is a database that stores RDF triples. Connecting data in a graph like manner. 


But using multiple datasets in a triple store is not a trivial task. If a user wants to query data from multiple datasets, they might not need all the data from each dataset. Instead, they wish to only query the data that is relevant to them. Knowing what data is relevant to a user can be a difficult task, depending on the dataset and the user.
This paper will focus on how the voiD(Vocabulary of Interlinked Datasets) vocabulary can be used to describe RDF datasets. 


There are many ways to do this, but for this research paper we will focus on the voiD vocabulary.


%%%%%%%%
We would like to describe RDF endpoints.
We would like to describe it in a manner that can effectively be used to describe federated queries.
We would like to make it simple for a machine to understand what a dataset contains, to make it easier for them to query relevant data.
We would like it to be used by a publisher, to describe the dataset in a way that makes it possible for a user to use a machine to extract important information from the data for the machine to make descicions about whether or not to construct a query, and how that query should look.    
----
How do we handle changes in the data? How do we handle changes in the voiD description?
How do ensure that all publishers live up to a common standard?
    - Should that be auto generated, used on a database?


----    
Find different IRI standards(for RDF datasets): Lars will add this
    -
Types of RDF endpoints: Lars will add this
    - SPARQL endpoint
        - Mentioned often

Difficulties in federated queries for RDF data: Gustav will add this
    - 

Compare other approaches for RDF data: Sebastian will add this
    - 



%For now, it seems that voiD(Vocabulary of Interlinked Datasets) is a good option for us to work with RDF datasets. 

% here is a link to the voiD description: http://www.w3.org/TR/void/
% https://ceur-ws.org/Vol-538/ldow2009_paper20.pdf